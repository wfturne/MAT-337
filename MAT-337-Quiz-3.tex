\documentclass[12pt,answers]{exam}
\usepackage[margin=1in]{geometry} 
\usepackage{amsmath,amsthm,amssymb,amsfonts,tikz,dcolumn,enumitem,fp,stmaryrd,rotating,etoolbox,cases}
\usetikzlibrary{decorations.markings}
 
\newcommand{\N}{\mathbb{N}}
\newcommand{\Z}{\mathbb{Z}}
\newcommand{\F}{\mathbb{F}}
\newcommand{\R}{\mathbb{R}}
\newcolumntype{2}{D{.}{}{2.0}}
 
\newenvironment{problem}[2][Problem]{\begin{trivlist}
\item[\hskip \labelsep {\bfseries #1}\hskip \labelsep {\bfseries #2.}]}{\end{trivlist}}
\setlist[enumerate]{itemsep=0mm}
\pagestyle{empty}
\errorcontextlines 10000

\AtBeginEnvironment{align}{\setcounter{equation}{0}}
\AtBeginEnvironment{numcases}{\setcounter{equation}{0}}
\begin{document}

\title{MAT 337 Quiz 3} 
\author{Billy Turner}
\maketitle
\thispagestyle{empty}

\begin{problem}{1}
State \textbf{precisely} and \textbf{completely} the definition of the dimension of a vector space $V$ over a field $\mathbb{F}$.
\end{problem}

\begin{solution}
Let $V$ be a vector space over $\mathbb{F}$ which can be generated by a finite set. We define the dimension of $V$ as the number of vectors in a basis of $V$. We denote it by dim$V$.
\end{solution}

\begin{problem}{2}
Is $\{1+2x+x^2, 3+x^2, x+x^2\}$ a basis of $P_{2}(\mathbb{R},x)$? Justify your answer.
\end{problem}

\begin{solution}
We claim that $S:=\{1+2x+x^2, 3+x^2, x+x^2\}$ is a basis of $V:=P_{2}(\mathbb{R},x)$. We will begin by showing that $S$ is linearly independent. Consider
\begin{align*}
	c_{1}(1+2x+x^2)+c_{2}(3+x^2)+c_{3}(x+x^2)=0+0x+0x^2
\end{align*}
for some $c_{1},c_{2},c_{3}\in\mathbb{R}$. By the definition of scalar multiplication of $V$, we observe
\begin{align*}
	c_{1}+2c_{1}x+c_{1}x^2+3c_{2}+c_{2}x^2+c_{3}x+c_{3}x^2=0+0x+0x^2
\end{align*}
and by the definition of addition of $V$,
\begin{align*}
	(c_{1}+3c_{2})+(2c_{1}+c_{3})x+(c_{1}+c_{2}+c_{3})x^2=0+0x+0x^2.
\end{align*}
By the definition of the equality of elements in $V$, we can observe the following system of equations:
\begin{numcases}
\\
	c_{1}+3c_{2}=0 \\
	2c_{1}+c_{3}=0 \\
	c_{1}+c_{2}+c_{2}=0 
\end{numcases}
By (1), we observe that $c_{2}=\frac{-c_{1}}{3}$ (4). By (2), we observe that $c_{3}=-2c_{1}$ (5). Substituting (4) and (5) into (3), we observe
\begin{align*}
	c_{1}+(\frac{-c_{1}}{3})+(-2c_{1})=0,
\end{align*}
which implies that $c_{1}=0$. Further, by (4) and (5), this implies that $c_{2}=0$ and $c_{3}=0$, respectively. Thus, $c_{1}=c_{2}=c_{3}=0$, and by the Highway for Linear Independence, S is linearly independent. Now consider $|S|=3=$ dim$V$. Since $S$ is a linearly independent subset of $V$ with $|S|$ = dim$V$, $S$ is a basis for $V$ by the theorem which states that a linearly independent subset of a vector space with cardinality equal to the dimension of the vector space is a basis for the vector space. 
\end{solution}

\begin{problem}{3}
Let $V$ be a vector space over a field $\mathbb{F}$ and $S=\{\vec{u},\vec{v}\}$ a generating subset of $V$. Show \textbf{directly} the following special case of the Replacement Theorem: \textit{for each $\vec{0}_V\neq \vec{w}\in V$, either $\{\vec{w},\vec{v}\}$ or $\{\vec{u},\vec{w}\}$ is a generating subset of $V$.}
\end{problem}
\begin{solution}
Let $V$ be a vector space over $\mathbb{F}$ and $S=\{\vec{u},\vec{v}\}$ a generating subset of $V$. Consider any $\vec{w}\in V$ such that $\vec{w}\neq \vec{0}_V$. Since $S$ generates $V$, this implies that Span$S$ = $V$. In addition, $\vec{w}\in V$, so we observe
\begin{align}
	\vec{w}=c_{1}\vec{u}+c_2\vec{v}
\end{align}
for some $c_{1},c_{2}\in \F$. Since $\vec{w}\neq \vec{0}_V$, it follows that not both of $c_{1}$ and $c_{2}$ are equal to zero. By relabeling $\vec{u}$ and $\vec{v}$ if necessary, assume that $c_{1}\neq 0$. Now observe by (1) and the definition of $\vec{0}_V$,
\begin{align*}
	\vec{0}_V+\vec{w}&=c_{1}\vec{u}+c_2\vec{v}.
\end{align*} Now by the definition of additive inverse of $c_2\vec{v}$ and the Cancellation Law, we observe
\begin{align*}
	c_2\vec{v}+(-c_2\vec{v})+\vec{w}&=c_{1}\vec{u}+c_2\vec{v} \\
	(-c_2\vec{v})+\vec{w}&=c_{1}\vec{u}.
\end{align*} By the commutativity of addition of $V$, we observe
\begin{align*}
	c_{1}\vec{u}&=\vec{w}+(-c_2\vec{v}) \\
	\vec{u}&=c^{-1}_{1}(\vec{w}+(-c_2\vec{v})).
\end{align*} Now by the Distribution Law \#1 and associativity of scalar multiplication of $V$,
\AtBeginEnvironment{align}{\setcounter{equation}{1}}
\begin{align}
	\vec{u}&=c^{-1}_{1}\vec{w}+(-c^{-1}_{1}c_2)\vec{v}.
\end{align} Let $S_1=\{\vec{w},\vec{v}\}$. We claim that $S_1$ generates $V$ or Span$S_1$ = Span$S$ = $V$. Since $\vec{w},\vec{v}\in V$, thus Span$S_1\subseteq V$ by the closure of addition and scalar multiplication of $V$. By (2), $\vec{u}\in$ Span$S_{1}$, and $\vec{v}\in$ Span$S_{1}$ since $1\cdot \vec{v}=\vec{v}$. Thus, $S\subseteq$ Span$S_1$, which implies that Span$S\subseteq$ Span$S_{1}$ by the theorem which states that the span of a subset of a vector space is a subspace and consequently is a subset of the vector space as well. Thus, Span$S_1$ = Span$S$ = $V$ by the Highway for Set Equality and $S_1=\{\vec{w},\vec{v}\}$ generates $V$. In the case that $c_1\neq0$, exchange $\vec{u}$ and $\vec{v}$ in the argument above and observe that $\{\vec{u},\vec{w}\}$ is a generating set of $V$. 
	




















\end{solution}


\end{document}