\documentclass[12pt,answers]{exam}
\usepackage[margin=1in]{geometry} 
\usepackage{amsmath,amsthm,amssymb,amsfonts,tikz,dcolumn,enumitem,fp,stmaryrd,rotating,etoolbox,cases}
\usetikzlibrary{decorations.markings}
 
\newcommand{\N}{\mathbb{N}}
\newcommand{\Z}{\mathbb{Z}}
\newcolumntype{2}{D{.}{}{2.0}}
 
\newenvironment{problem}[2][Problem]{\begin{trivlist}
\item[\hskip \labelsep {\bfseries #1}\hskip \labelsep {\bfseries #2.}]}{\end{trivlist}}
\setlist[enumerate]{itemsep=0mm}
\pagestyle{empty}
\errorcontextlines 10000

\AtBeginEnvironment{align}{\setcounter{equation}{0}}
\AtBeginEnvironment{numcases}{\setcounter{equation}{0}}
\begin{document}

\title{MAT 337 Quiz 2} 
\author{Billy Turner}
\maketitle
\thispagestyle{empty}

\begin{problem}{1}
State \textbf{precisely} and \textbf{completely} the definition of the following:
\begin{enumerate}[label=(\alph*)]
\item a function $f$ from a set $A$ to a set $B$;
\item a subspace $W$ of a vector space $V$ over a field $\mathbb{F}$;
\item a linear combination of vectors of a subset $S$ of a vector space $V$ over a field $\mathbb{F}$;
\item Span$S$ of a subset $S$ of a vector space $V$ over a field $\mathbb{F}$.
\end{enumerate}
\end{problem}

\begin{solution}
\begin{enumerate}[label=(\alph*)]
\item A \textbf{function} $f$ from a set $A$ to a set $B$ is a rule which assigns to every element of $A$ a unique element of $B$. We denote $f$ by
\begin{align*}
f:A\rightarrow B
\end{align*}
and call $A$ the domain of $f$ and B the codomain of $f$.
\item Let $V$ be a vector space over $\mathbb{F}$ and $W$ a non-empty subset of $V$. We call $W$ a \textbf{subspace} of $V$ if $W$ with the operations of addition and scalar multiplication is a vector space over $\mathbb{F}$.
\item Let $V$ be a vector space over $\mathbb{F}$ and $S$ a non-empty subset of $V$. A vector $\vec{u}$ is called a \textbf{linear combination} of vectors of $S$ if
\begin{align*}
\vec{u}=c_{1}\vec{v}_{1}+c_{2}\vec{v}_{2}+...+c_{n}\vec{v}_{n}
\end{align*}
for some $v_{i}'s\in S$ and $c_{i}'s\in \mathbb{F}$. 
\item Let $V$ be a vector space over $\mathbb{F}$ and $S$ a non-empty subset of $V$. We denote by Span$S$ the set of all linear combinations of vectors of $S$. 
\end{enumerate}
\end{solution}

\begin{problem}{2}
Decide if the following given subset $W$ is a vector subspace of the given vector space $V$. Justify your answers.
\begin{enumerate}[label=(\alph*)]
\item $V=\mathbb{R}^{3}$ and $W$ is the subset of all the elements $(a,b,c)\in V$ such that $2b=3c$.
\item $V=\mathbb{R}^{3}$ and $W$ is the subset of all the elements $(a,b,c)\in V$ such that $a^{2}-bc=0$.
\item $V= \mathcal{F}(\mathbb{R},\mathbb{R})$ and $W$ is the subset of all the elements $f\in V$ such that $f(1)+5f(2)=0$.
\end{enumerate}
\end{problem}

\begin{solution}
\begin{enumerate}[label=(\alph*)]
\item Let $V=\mathbb{R}^{3}$ and $W$ be the subset of all the elements $(a,b,c)\in V$ such that $2b=3c$. 

	\underline{\textit{1st}}: The zero vector $\vec{0}_{v}=(0,0,0)\in W$ by the definition of the subset $W$.
	
	\underline{\textit{2nd}}: For every $\vec{u},\vec{v} \in W$, we wish to show that $\vec{u}+\vec{v}\in W$. Consider $\vec{u}=(a_1,b_1,c_1)$ and $\vec{v}=(a_2,b_2,c_2)$ for some 		$a_1,a_2,b_1,b_2,c_1,c_2\in \mathbb{R}$. In addition, based on the definition of $W$, we observe
	\begin{align}
		2b_1 &= 3c_1\text{, and} \\
		2b_2 &= 3c_2.
	\end{align}
	Now consider the sum of $\vec{u}$ and $\vec{v}$
	\begin{align*}
		\vec{u}+\vec{v} &= (a_1,b_1,c_1) + (a_2,b_2,c_2) \\
		&= (a_1+a_2,b_1+b_2,c_1+c_2),
	\end{align*}
	by the definition of the addition of elements of $W$. To show that $\vec{u}+\vec{v} \in W$, we observe
	\begin{align*}
		2(b_1+b_2) &= 2b_1+2b_2 \text{, by Distribution Law \#1 of $W$,} \\
		&= 3c_1+3c_2 \text{, by (1) and (2) above,} \\
		&= 3(c_1+c_2) \text{, by Distribution Law \#1 of $W$.}
	\end{align*}
	Since $2(b_1+b_2)=3(c_1+c_2)$, thus $\vec{u}+\vec{v} \in W$. 
	
	\underline{\textit{3rd}}: For every $\vec{u} \in W$ and $\alpha\in \mathbb{R}$, we wish to show that $\alpha\vec{u}\in W$. Consider $\vec{u}=(a,b,c)$ for some $a,b,c\in W$. In 			addition, based on the definition of $W$, we observe
	\begin{align}
		2b &= 3c.
	\end{align}
	Now consider,
	\begin{align*}
		\alpha\vec{u} &= \alpha(a,b,c) \\
		&= (\alpha a,\alpha b, \alpha c),
	\end{align*}
	by the definition of scalar multiplication of $W$. To show that $\alpha\vec{u}\in V$, we observe,
	\begin{align*}
		2(\alpha b &)= \alpha (2b) \text{, by the Commutative Property of $\mathbb{R}$,} \\
		&= \alpha (3c) \text{, by (1) above} \\
		&= 3(\alpha c) \text{, by the Commutative Property of $\mathbb{R}$.}
	\end{align*}
	Since $2(\alpha b) = 3(\alpha c$), thus $\alpha\vec{u}\in W$.
	
	Finally, by the Highway for subspaces, $W$ is a subspace of $V$. 
	
\item $W$ as defined in (b) above is not a subspace of $V=\mathbb{R}^{3}$. Consider $(1,1,1),(4,2,8)\in W$ by the definition of $W$. However,
	\begin{align*}
		(1,1,1)+(4,2,8)=(5,3,9)\notin W
	\end{align*}
	since $5^2-(3)(9)\neq0$. Thus, by the Highway for subspaces, $W$ is not a subspace of $V$.
	
\item \underline{\textit{1st}}: The zero vector 
	\begin{align*}
		\vec{0}_{v}: \mathbb{R} \rightarrow \mathbb{R}
	\end{align*}
	is in $W$ since 
	\begin{align*}
		\vec{0}_{v}(1)+5\vec{0}_{v}(2)=0+0=0.
	\end{align*}
	
	\underline{\textit{2nd}}: For every $f,g \in W$, we wish to show that $f+g\in W$. Based on the definition of $W$, we observe,
	\begin{align}
		f(1)+5f(2) &= 0 \text{, and} \\
		g(1)+5g(2) &= 0	
	\end{align}
	In addition, for every $t\in\mathbb{R}$, we observe
	\begin{align*}
		(f+g)(t)=f(t)+g(t)
	\end{align*}
	by the definition of addition of $W$. 
	To show that $f+g\in W$, we observe
	\begin{footnotesize}
	\begin{align*}
		(f+g)(1)+5(f+g)(2) &= (f(1)+g(1))+5(f(2)+g(2)) \text{, by the definition of addition of $W$,} \\
		&= f(1)+g(1)+5f(2)+5g(2) \text{, by the Distributive Law \#1 of $W$, } \\
		&= (f(1)+5f(2))+(g(1)+5g(2)) \text{, by the Associative and } \\ &\hspace{5.5 cm}{\text{Commutative properties of $\mathbb{R}$,}} \\
		&= 0 + 0 \text{, by (1) and (2) above,} \\
		&=0.
	\end{align*}
	\end{footnotesize}
	Since $(f+g)(1)+5(f+g)(2)=0$, thus, $f+g\in W$. 
	
	\underline{\textit{3rd}}: For every $f\in W$ and $\alpha\in \mathbb{R}$, we wish to show that $\alpha f\in W$. Based on the definition of $W$, we observe
	\begin{align}
		f(1)+5f(2) &= 0
	\end{align}
	In addition, for every $t\in\mathbb{R}$, we observe
	\begin{align*}
		(\alpha f)(t)=\alpha f(t),
	\end{align*}
	by the definition of scalar multiplication of $W$. To show that $\alpha f\in V$, we observe,
	\begin{small}
	\begin{align*}
		(\alpha f)(1)+5(\alpha f)(2) &= \alpha f(1)+5(\alpha f(2)) \text{, by the definition of scalar multiplication of $W$,} \\
		&= \alpha f(1)+\alpha(5f(2)) \text{, by the Associate property of multiplication of $\mathbb{R}$,} \\
		&= \alpha (f(1)+5f(2)) \text{, by the Distributive Law \#1 of $\mathbb{R}$,} \\
		&= \alpha (0) \text{, by (1) above,} \\
		&= 0.
	\end{align*}
	\end{small}
	Since $(\alpha f)(1)+5(\alpha f)(2)=0$, thus $\alpha f \in W$.
	
	Finally, by the Highway for subspaces, $W$ is a subspace of $V$. 
\end{enumerate}
\end{solution}

\begin{problem}{3}
Problems 5b) and 5g) on page 34. \textbf{Justify your answers}. 
\end{problem}

\begin{solution}

\noindent \textbf{5b)} Given the vector $(-1,2,1)$ and the set $S=\{(1,0,2), (-1,1,1)\}$, we wish to determine if $(-1,2,1)\in \text{Span}S$. If $(-1,2,1)\in \text{Span}S$, then it can be written as a linear combination of vectors of $S$, or 
\begin{align*}
	(-1,2,1) &= c_{1}(1,0,-2)+c_{2}(-1,1,1)
\end{align*}
for some $c_{1},c_{2}\in \mathbb{R}$. Now, we observe
\begin{align*}
	(-1,2,1) &= (c_{1}, 0, -2c_{1})+(-c_{2},c_{2},c_{2}), \\
	(-1,2,1) &= (c_{1}-c_{2}, c_{2}, -2c_{1}+c_{2}).
\end{align*}
From this linear combination, we can generate a system of linear equations as follows:
\begin{numcases}
\\
-1 = c_{1}-c_{2} \\
2 = c_{2} \\
1 = -2c_{1}+c_{2}
\end{numcases}
Substituting (2) into (1), we observe $-1=c_{1}-2$ which implies that $c_{1}=1$. In addition, substituting (2) into (3) yields $1=-2c_{1}+2$, which implies that $c_{1}=\frac{1}{2}$. Thus, $1=\frac{1}{2}$, which is a contradiction. Thus, $(-1,2,1)\notin \text{Span}S$.

\noindent \textbf{5b)} Given the vector $\Bigl( \begin{matrix}1 & 2\\ -3 & 4\end{matrix}\Bigr)$, and the set $S=\Bigl\{ \Bigl( \begin{matrix}1 & 0\\ -1 & 0\end{matrix}\Bigr), \Bigl( \begin{matrix}0 & 1\\ 0 & 1\end{matrix}\Bigr), \Bigl( \begin{matrix}1 & 1\\ 0 & 0\end{matrix}\Bigr) \Bigr\}$, we wish to determine if $\Bigl( \begin{matrix}1 & 2\\ -3 & 4\end{matrix}\Bigr)\in \text{Span}S$. If $\Bigl( \begin{matrix}1 & 2\\ -3 & 4\end{matrix}\Bigr)\in \text{Span}S$, then it can be written as a linear combination of vectors of $S$, or
\begin{align*}
	\Bigl( \begin{matrix}1 & 2\\ -3 & 4\end{matrix}\Bigr)=c_{1}\Bigl( \begin{matrix}1 & 0\\ -1 & 0\end{matrix}\Bigr)+c_{2}\Bigl( \begin{matrix}0 & 1\\ 0 & 1\end{matrix}\Bigr)					+c_{3}\Bigl( \begin{matrix}1 & 1\\ 0 & 0\end{matrix}\Bigr)
\end{align*}
for some $c_{1},c_{2},c_{3}\in \mathbb{R}$. Now, we observe
\begin{align*}
	\Bigl( \begin{matrix}1 & 2\\ -3 & 4\end{matrix}\Bigr) &= \Bigl( \begin{matrix}c_{1} & 0\\ -c_{1} & 0\end{matrix}\Bigr)+\Bigl( \begin{matrix}0 & c_{2}\\ 0 & c_{2}\end{matrix}\Bigr)					+\Bigl( \begin{matrix}c_{3} & c_{3}\\ 0 & 0\end{matrix}\Bigr) \\
	\Bigl( \begin{matrix}1 & 2\\ -3 & 4\end{matrix}\Bigr) &= \Bigl( \begin{matrix} c_{1}+c_{3} & c_{2}+c_{3}\\ -c_{1} & c_{2}\end{matrix}\Bigr). 
\end{align*}
From this linear combination, we can generate a system of linear equations as follows:
\begin{numcases}
\\
1 = c_{1}+c_{3} \\
2 = c_{2}+c_{3}  \\
-3 = -c_{1} \\
4 = c_{2}
\end{numcases}
From this system, we observe that $c_{1}=3$ and $c_{2}=4$. Substituting (3) into (1), we observe $1=3+c_{3}$, which implies that $c_{3}=-2$. Substituting (4) into (2), we observe $2=4+c_{3}$, which also implies that $c_{3}=-2$. Thus, a solution to system exists, which implies that
\begin{align*}
	 \Bigl( \begin{matrix}1 & 2\\ -3 & 4\end{matrix}\Bigr)=3\Bigl( \begin{matrix}1 & 0\\ -1 & 0\end{matrix}\Bigr)+4\Bigl( \begin{matrix}0 & 1\\ 0 & 1\end{matrix}\Bigr)						+-2\Bigl( \begin{matrix}1 & 1\\ 0 & 0\end{matrix}\Bigr).
\end{align*}
Finally, since  $\Bigl( \begin{matrix}1 & 2\\ -3 & 4\end{matrix}\Bigr)$ can be written as a linear combination of vectors of $S$, thus $\Bigl( \begin{matrix}1 & 2\\ -3 & 4\end{matrix}\Bigr)\in \text{Span}S$.
\end{solution}


\end{document}q